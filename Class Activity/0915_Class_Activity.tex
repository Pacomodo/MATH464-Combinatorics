\documentclass[a4paper, 12pt]{article}
\usepackage[a4paper, total={6in, 9.5in}]{geometry} % DL: 쪽여백 조정
\usepackage{setspace} % DL: 줄간격 조정
\usepackage{graphicx} % Required for inserting images
\usepackage{kotex}
\usepackage{amsmath,amsthm,amssymb,amsfonts,mdframed}

% set indent length to zero.
\setlength{\parindent}{0pt}

\title{0915 Class Activity}
\author{박예영}

\begin{document}
\maketitle
\begin{mdframed}

Parking function에 대해, 

(1) 주차장소에 대한 선호도를 갖는 n대의 자동차가 일렬로 늘어선 n개의 주차공간에 주차하는 방법으로 정의한 것과

(2) 수열에 대한 조건으로 정의한 것이

왜 일치하는 지 설명하시오.

\end{mdframed}
\doublespacing

\begin{proof}


($\Rightarrow$)
우리는 (1)번 정의를 만족시킨다면, (2)번 정의를 만족시킴을 보이면 됩니다. 즉, (2)번 정의를 만족시키지 않는 모든 수열은 (1)번 정의 또한 만족되지 않음을 보입시다.

먼저, (2)번 정의를 만족시키지 않는 수열 $(a_1, a_2, ... , a_n)$을 생각해 봅시다. 이 수열을 단조 증가하는 수열로 재배열 시켰을 때의 수열을 $(b_1, b_2, ... , b_n)$이라고 합시다. (2)번 정의를 만족시키지 않으므로, $(b_1, b_2, ..., b_n)$의 어떤 한 원소 $b_i$는 $i$보다 크다고 해봅시다.

$(b_1, b_2, ... , b_n)$은 단조 증가하는 수열이라고 했으므로, $b_i$뒤에 있는 $b_{i+1}, b_{i+2}, ... , b_n$은 모두 $i$보다 큽니다. 다시 말해, $b_i \sim b_n$은 모두 $i$보다 큽니다.

즉, $(a_1, a_2, ..., a_n)$ 중에서 $i$보다 큰 원소가 적어도 $n-i+1$개 있음을 알 수 있습니다.

(1)번 정의에 의하면 $(a_1, a_2, ... , a_n)$은 각 차량이 원하는 주차공간 번호를 의미하는데, $i$번보다 큰 주차공간 번호를 원하는 차량이 적어도 $n-i+1$대 있다는 뜻으로 해석할 수 있습니다.

이 $n-1+1$대의 차량은 모두 $i$번보다 큰 번호를 선호하므로, $i$번 주차공간 뒤, 다시 말해, $(i+1) \sim n$번의 주차공간에 주차를 할 수 밖에 없습니다.

가능한 주차공간은 $n-i$개가 있고, $n-i+1$대의 차량이 주차를 하기에는 모자라므로 (1)번 정의를 만족하지 못함을 알 수 있습니다.


($\Leftarrow$)
(2)번 정의를 만족시킨다면, (1)번 정의를 만족시킴을 보입시다. 즉, (1)번 정의를 만족시키지 못하면 (2)번 정의또한 만족시키지 못함을 보입시다.

그 말은 곧, $i$번 차가 들어왔는데 이 차가 원하는 주차공간의 번호부터 뒤에 있는 모든 주차공간이 가득 차서 더 이상 주차를 할 수 없다는 뜻이 됩니다.

각 차량 별로 원하는 주차공간 번호를 나타낸 수열 $(a_1, a_2, ... , a_n)$을 생각합시다. $i$번째 차량이 주차를 하려고 한다면, 그 이전에 $a_i \sim n$번의 주차공간이 모두 가득 차 있어야 합니다. 우리는 이 수열 $(a_1, a_2, ... , a_n)$를 단조 증가하도록 재배열한 수열 $(b_1, b_2, ... , b_n)$를 생각했을 때, 어떤 $b_j > j$임을 보이면 됩니다.

$i$번째 차량이 들어오기 전에 상황을 살펴봅시다. $(i-1)$대의 차량은 모두 잘 주차되어있고, 이때, $a_i \sim n$번의 주차공간은 모두 가득 차 있습니다. $(i-1)$번째 차량까지 주차를 했을 때 $p \sim n$번의 주차공간이 가득 차 있다고 하면, 이러한 $p$ 중에서 가장 작은 것을 $m$이라고 합시다.

먼저, $m$은 1이 될 수 없습니다. $m = 1$이라고 하면 $i$번째 차량이 주차하기 이전부터 $1 \sim n$번의 주차공간이 모두 가득 차 있다는 뜻이 되는데, 이는 차량과 주차공간의 수가 맞지 않으므로 말이 되지 않습니다.

따라서 $m>1$이라고 해봅시다. 즉, $(i-1)$번째 차량까지 주차를 했을 때, $m \sim n$번의 주차공간은 가득 차 있고, $(m-1)$번의 주차공간은 비어있습니다.

$1 \sim (i-1)$번째 차량들 중에서 $m$번 주차공간 이후에 주차한 차량들이 원했던 주차공간의 번호들은 모두 $m$보다 크거나 같습니다. 만약에 그러지 않았더라면, 즉, $m$번 주차공간 이후에 주차했던 차량이 원했던 주차공간의 번호가 $m$보다 작았더라면, 아무리 밀려나도 $(m-1)$번 주차공간에 주차를 할 수 있었으므로 말이 안됩니다.

$m \sim n$번 주차공간이 가득 차 있었으므로, 적어도 $n-m+1$대의 차량이 원했던 주차공간의 번호는 모두 $m$보다 크거나 같습니다. 우리는 $a_i$가 $m$보다 크거나 같다는 사실을 알고 있습니다. 즉, 우리는 적어도 $n-m+2$대의 차량($i$번째 차량을 포함하여)이 $m$보다 크거나 같다는 사실을 알고 있습니다.

이제, $(a_1, a_2, ... , a_n)$을 단조 증가하도록 정렬한 $(b_1, b_2, ... , b_n)$을 봅시다. $b_{m-1}$을 봅시다. 이는 $(a_1, a_2, ... , a_n)$ 중에서 $n-m+2$번째로 큰 수이고, 우리는 $m$보다 크거나 같은 $a_i$가 적어도 $n-m+2$개 있다는 사실을 알고 있으므로, $b_{m-1}$ 또한 $m$보다 크거나 같음을 알 수 있습니다.

따라서 이는 (2)번 정의를 만족하지 않는 수열임을 알 수 있습니다.

\end{proof}

\end{document}
