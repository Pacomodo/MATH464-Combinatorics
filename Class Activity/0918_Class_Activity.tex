\documentclass[a4paper, 12pt]{article}
\usepackage[a4paper, total={6in, 9.5in}]{geometry} % DL: 쪽여백 조정
\usepackage{setspace} % DL: 줄간격 조정
\usepackage{graphicx} % Required for inserting images
\usepackage{kotex}
\usepackage{amsmath,amsthm,amssymb,amsfonts,mdframed}

% set indent length to zero.
\setlength{\parindent}{0pt}

\title{0918 Class Activity}
\author{박예영}

\begin{document}
\maketitle
\begin{mdframed}
다음과 같이 정의된 수열 {$a_n$}의 일반항을 생성함수(generating function)방법을 이용하여 구해봅시다.

$$
a_n - 3 a_{n-1} = n^2 \textrm{ for }n>0 \textrm{ and } a_0 = 1
$$
\end{mdframed}
\doublespacing

\begin{proof}

생성 함수 $\displaystyle f(x) = \sum_{n=0}^{\infty} a_n x^n$를 생각해 봅시다.

Recurrence relation을 활용하기 위해, $\displaystyle \sum_{n=1}^{\infty} (a_n - 3a_{n-1} ) x^n$을 생각해봅시다.



이 식은 다음과 같이 변형시킬 수 있습니다.
\begin{align*}
    \sum_{n=1}^{\infty} (a_n - 3a_{n-1} ) x^n & = \sum_{n=1}^{\infty} n^2 x^n \\
    & = \sum_{n=1}^{\infty} (n^2 - n)x^n + nx^n \\
    & = \sum_{n=1}^{\infty} n(n-1)x^n + \sum_{n=1}^{\infty} nx^n \\
    & = x^2 \sum_{n=2}^{\infty} n(n-1)x^{n-2} + x \sum_{n=1}^{\infty} nx^{n-1}
\end{align*}
$\displaystyle g(x) = \sum_{n=0}^{\infty}x^n = \frac{1}{1-x}$이므로, 이를 이용하면 다음과 같이 쓸 수 있습니다.
\begin{align*}
    x^2 \sum_{n=2}^{\infty} n(n-1)x^{n-2} + x \sum_{n=1}^{\infty} nx^{n-1} & = x^2 g^{''} + xg^{'} \\
    & = \frac{2x^2}{(1-x)^3} + \frac{x}{(1-x)^2} \\
    & = \frac{x^2 + x}{(1-x)^3}
\end{align*}
한편, 생성 함수 $f(x)$를 활용하여 같은 식을 다르게 변형시킬 수 있습니다.
\begin{align*}
    \sum_{n=1}^{\infty} (a_n - 3a_{n-1} ) x^n & = \sum_{n=1}^{\infty} a_n x^n -3x\sum_{n=1}^{\infty} a_{n-1} x^{n-1} \\
    & = f(x) - 1 - 3xf(x) \\
    & = f(x)(1-3x) - 1
\end{align*}
따라서, $f(x)$는 다음과 같습니다.
\begin{align*}
    \frac{x^2 + x}{(1-x)^3} = f(x)(1-3x) - 1 \\
    \Rightarrow f(x) = \frac{x^2 + x}{(1-x)^3(1-3x)} + \frac{1}{1-3x}
\end{align*}
우리는 $a_n$을 구하기 위해 $f(x)$를 다시 무한합 꼴로 고쳐야합니다. Geometric series를 이용하면 무한합 꼴로 쉽게 고칠 수 있으므로, 부분분수 분해를 통해 원하는 계수를 얻어내봅시다.
\begin{align*}
    f(x) & = \frac{x^2 + x}{(1-x)^3(1-3x)} + \frac{1}{1-3x} \\
    & = \frac{a}{1-x} + \frac{b}{(1-x)^2} + \frac{c}{(1-x)^3} + \frac{d}{1-3x} + \frac{1}{1-3x}
\end{align*}
이렇게 놓고, 계수 $a, b, c, d$를 구해봅시다.
\begin{align*}
    \textrm{즉, }x^2 + x & = a(1-x)^2 (1-3x) \\
    & + b(1-x)(1-3x) \\
    & + c(1-3x) \\
    & + d(1-x)^3
\end{align*}
를 만족시키는 계수 $a, b, c, d$를 구해봅시다.

먼저, 위의 식에 $x=1$을 대입하면 $c=-1$임을 쉽게 알 수 있습니다.

또한, 위의 식에 $x=1/3$을 대입하면 $d=3/2$임을 알 수 있습니다.

이후, $x=0$을 대입하면 $a+b=-1/2$라는 식을 얻어낼 수 있습니다.

$x=2$를 대입하면 $6 = -5a + 5b + 5 - 3/2 \rightarrow 5a - 5b = -5/2 \rightarrow a-b = -1/2$를 얻어낼 수 있습니다.

따라서, $(a, b, c, d) = (-1/2, 0, -1, 3/2)$입니다.

이를 참고하여 $f(x)$를 고치면 다음과 같습니다.
\begin{align*}
    f(x) & = \frac{a}{1-x} + \frac{b}{(1-x)^2} + \frac{c}{(1-x)^3} + \frac{d}{1-3x} + \frac{1}{1-3x} \\
    & = - \frac{1}{2(1-x)} - \frac{2}{2(1-x)^3} + \frac{5}{2(1-3x)} \\
    & = - \frac{1}{2} \sum_{n=0}^{\infty} x^n - \frac{1}{2} \sum_{n=2}^{\infty} n(n-1) x^{n-2} + \frac{5}{2} \sum_{n=0}^{\infty} (3x)^n \\
    & = \sum_{n=0}^{\infty} \left( - \frac{1}{2} - \frac{(n+1)(n+2)}{2} + \frac{5}{2}3^n \right) x^n \\
    & = \sum_{n=0}^{\infty} a_n x^n
\end{align*}

따라서 $\displaystyle a_n = - \frac{1}{2} - \frac{(n+1)(n+2)}{2} + \frac{5}{2}3^n = \frac{5\cdot3^n - n^2 - 3n - 3}{2}$입니다.
\end{proof}

\end{document}