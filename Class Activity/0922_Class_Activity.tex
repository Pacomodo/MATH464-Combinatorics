\documentclass[a4paper, 12pt]{article}
\usepackage[a4paper, total={6in, 9.5in}]{geometry} % DL: 쪽여백 조정
\usepackage{setspace} % DL: 줄간격 조정
\usepackage{graphicx} % Required for inserting images
\usepackage{kotex}
\usepackage{amsmath,amsthm,amssymb,amsfonts,mdframed}

% set indent length to zero.
\setlength{\parindent}{0pt}

\title{0922 Class Activity}
\author{박예영}

\begin{document}
\maketitle
\begin{mdframed}
Show that if $G$ is a simple graph with $p$ vertices, where each vertex has degree not less than $\displaystyle \frac{p-1}{2}$, then $G$ must be connected.
\end{mdframed}
\doublespacing

\begin{proof}
Suppose that $G$ is not connected simple graph with $p$ vertices and each vertex in $G$ has degree not less than $\displaystyle \frac{p-1}{2}$. Then, we can split the vertex set $V$ into two sets $V_1$ and $V_2$, where for any $v \in V_1$ and $w \in V_2$, there are no edges $(v, w)$ in the edge set $E$.

Let $|V_1| = a$ and $|V_2| = b$. Since each vertex in $V_1$ has degree not less than $\displaystyle \frac{p-1}{2}$ and these vertices are not connected into $V_2$, the number of vertices in $V_1$, $a$, must be larger than or equal to $\displaystyle \frac{p-1}{2} + 1$.

Similarly, the number of vertices in $V_2$, $b$, must be larger than or equal to $\displaystyle \frac{p-1}{2} + 1$.

Hence, the number of vertices in $V$ must be larger than or equal to $\displaystyle (p-1) + 1 + 1 = p+1$, which contradicts to $G$ has $p$ vertices.

\end{proof}

\end{document}